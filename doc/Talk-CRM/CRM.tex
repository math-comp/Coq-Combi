\documentclass[compress,11pt]{beamer}

\usetheme{TALK}
\usepackage[vcentermath]{genyoungtabtikz}
\usepackage{minted}
\usemintedstyle{emacs}
%\usemintedstyle{colorful}
%\usemintedstyle{borland}
%\usemintedstyle{autumn}

\usepackage{genyoungtabtikz}

\newminted{coq}{
frame=lines,
framesep=2mm,
fontsize=\scriptsize,
mathescape=true
}
\usepackage{commath}

% INFO DOCUMENT - TITRE, AUTEUR, INSTITUTION
\title{\bf\LARGE A formal proof of the \\
Littlewood-Richardson rule\\[5mm]}
\author{Florent Hivert}
\institute[LRI]{
  LRI / Université Paris Sud 11 / CNRS}
\date[September 2018]{September 2018}

\newcommand{\XX}{{\mathbb X}}

\newcommand{\free}[1]{\left\langle#1\right\rangle}
\newcommand{\N}{{\mathbb N}}
\newcommand{\C}{{\mathbb C}}
\newcommand{\Q}{{\mathbb Q}}
\newcommand{\SG}{{\mathfrak S}}
\newcommand{\std}{\operatorname{Std}}

\newcommand{\sym}{\mathrm{sym}}
\newcommand{\NCSF}{\mathbf{NCSF}}
\newcommand{\QSym}{\mathrm{QSym}}
\newcommand{\FSym}{\mathbf{FSym}}

\newcommand{\partof}{\vdash}                    % Partition de
\newcommand{\compof}{\vDash}                    % Composisition de

\newcommand{\qandq}{\text{\quad et\quad}}

\newcommand{\alphX}{{\mathbb X}}
%%%%%%%%%%%%%%%%%%%%%
\newcommand{\red}[1]{{\color{red} #1}}
\newcommand{\green}[1]{{\color{red} #1}}
%------------------------------------------------------------------------------
\begin{document}

% PAGE D'ACCUEIL
\frame{\titlepage}

\begin{frame}{Outline}
  \tableofcontents
\end{frame}

\section{Motivation : certified proof in combinatorics}
%%%%%%%%%%%%%%%%%%%%%%%%%%%%%%%%%%%%%%%%%%%%%%%%%%%%%%%%
\begin{frame}{Why formalize things on computers}

  \begin{center}
    \textbf{\LARGE Writing correct programs is hard:}
  \end{center}
\medskip

\begin{itemize}
\item The human mind is focused on the big picture;
\item Hard to take track of all the trivial / particular cases.
\end{itemize}
\bigskip\pause

Some excerpts of my contribution to \texttt{Sagemath}:
\begin{itemize}
\item determinant / rank / invertibility of $0\times0$ and $1\times1$ matrices
\item empty set and its permutation
\item empty partition / composition / parking function / tableau \dots
\item the $0$ and $1$ species
\item \dots
\end{itemize}
\bigskip\pause

\centering{\textbf{\huge What about proofs ?}}
\end{frame}

\begin{frame}{Are our proofs always correct ?}

Donald Knuth:
\begin{quote}
  Beware of bugs in the above code; I have only proved it correct, not tried it.
\end{quote}
\bigskip\pause

Often in combinatorics, and particularly in \textbf{bijective} combinatorics,
proofs are algorithms, together with justifications that they meet their
specifications.
\end{frame}

\begin{frame}{Are our proofs always correct ?}

  The Littlewood-Richardson rule:

  \begin{itemize}
  \item stated (1934) by D. E. Littlewood and A. R. Richardson, \red{wrong
      proof, wrong example}.
  \item Robinson (1938), wrong completed proof.
  \item more wrong published proofs...
  \item first correct proof: Schützenberger (1977).
  \item nowadays: dozens of different proofs\dots
  \end{itemize}

  \begin{quotation}\small
    \textbf{Wikipedia}: The Littlewood–Richardson rule is \textbf{notorious
      for the number of errors} that appeared prior to its complete, published
    proof. Several published attempts to prove it are incomplete, and it is
    particularly difficult to avoid errors when doing hand calculations with
    it: even the original example in D. E. Littlewood and A. R. Richardson
    (1934) contains an error.
 \end{quotation}
\end{frame}


\begin{frame}{The case of the Littlewood-Richardson rule ?}

  A footnote in Macdonald's book:
  \medskip

  \begin{quotation}
    \noindent
    Gordon James reports that he was once told that:

    ``The Littlewood-Richardson rule helped to {\color{green} get men on the
      moon}, but it was {\color{red} not proved until after they had got
      there}. The first part of this story might be an exaggeration.''
  \end{quotation}
  \bigskip

  This sentence appears in James, G. D. (1987) \emph{The representation theory
    of the symmetric groups}.
\end{frame}

\begin{frame}{The case of the Littlewood-Richardson rule ?}
More quotation of James:

\begin{quotation}\small
  It seems that for a long time \textbf{the entire body of experts in the
    field was convinced} by these proofs; at any rate it was not until 1976
  that McConnell pointed out \textbf{a subtle ambiguity} in part of the
  construction underlying the argument.

  [...]

  How was it possible for an \red{incorrect proof} of such a central result in
  the theory of $S_n$ to have been \red{accepted for close to forty years}?
  The level of rigor customary among mathematicians when a combinatorial
  argument is required, is (\green{probably quite rightly}) of the
  \green{nonpedantic hand-waving} kind; perhaps one lesson to be drawn is that
  a \textbf{higher degree of care} will be needed in dealing with such
  combinatorial complexities as occur in the present level of development of
  Young's approach.
\end{quotation}
\end{frame}

\begin{frame}[fragile]
  \begin{problem}
    Suppose that, back in 1977, they had had our current proof assistant
    technology. Would it have been \textbf{feasible} to check Schützenberger
    proof ? If so, \textbf{how long} would it have taken ?
  \end{problem}
  \bigskip\pause

  \begin{Theorem}[Constructive answer !]
    Yes ! Less than 5 month and two weeks !
  \end{Theorem}

\small
\begin{verbatim}
commit f990146b8c6e062fe025740a08f888deb9481c2d
Date:   Thu Jul 24 17:46:58 2014 +0200
    Schensted's algorithm.

commit 2418282695455261e5459b33d3e8f979d57c3bdb
Date:   Sun Jan 4 15:31:16 2015 +0100
    DONE the proof of the Littlewood_Richardson rule !!!!
\end{verbatim}
\end{frame}

\section{A short introduction to formal proof in Coq/Mathcomp}
%%%%%%%%%%%%%%%%%%%%%%%%%%%%%%%%%%%%%%%%%%%%%%%%%%%%%%%%%%%%%%%%%
\begin{frame}{History of Coq and Mathcomp}

  \begin{itemize}
  \item 1985 -- T.~Coquand :  \emph{Calculus of constructions}
  \item 1989 -- T.~Coquand, G.~Huet: creation of \texttt{Coq}
  \item 2004 -- G.~Gonthier, B.~Werner : \emph{4 color theorem} in Coq

    Along their way \texttt{Ssreflect} ``small scale reflection''.

  \item 2006 -- 2018 Mathematical component: a library of formalized mathematics.
    \begin{itemize}
    \item basic data structures, algebra, group an representation theory;
    \item the infrastructure for the machine checked proofs of:
    \end{itemize}
  \item 2012 -- Coq checked proof of Feit-Thomson's theorem:

    Every finite group of odd order is solvable.
  \end{itemize}
\end{frame}

\begin{frame}{Formal (mechanized) proofs}

  \begin{block}{Aim}
    Write a proof that is checked by computer all the way down to the
    logical foundation.
  \end{block}
  \bigskip\pause

  \textbf{Proof assistant} (interactive theorem prover) : \\ A kind of Integrated
  Development Environment (IDE) which helps writing such proofs by constantly
  checking the coherence and keeping track of missing parts.  \bigskip

  Note: Proof assistant $\neq$ automated theorem prover
\end{frame}


\begin{frame}{What is needed to build a proof assistant ?}

Three ingredients:
\begin{enumerate}
\item A way to \textbf{\red{store algorithm}} that allows for\\
  \textbf{manipulating} them \textbf{and reasoning} about them;
\bigskip\pause

\item A way to \textbf{\red{store proofs}} that allows for\\
  \textbf{manipulating} them \textbf{and reasoning} about them;
\bigskip\pause

\item A way to \textbf{mechanically check} everything.
\end{enumerate}
\end{frame}

\begin{frame}{Proofs as programs (Curry-Howard)}

Suppose that
\begin{itemize}
\item we have \textbf{data} encoding a proof $a$ and two statements $A$ and
  $B$
\item the system is able to make so-called \textbf{judgments}: \\
  to verify that $a$ is a correct proof of $A$ (written as $a : B$)
\end{itemize}
\pause\medskip

Then, a statement $A \to B$ means that each time we have a proof of $A$, we
can construct a proof of $B$.
\pause\bigskip

\begin{block}{Curry-Howard correspondence in a nutshell}
  The idea is ``simply'' to encode a proof of $A \to B$ by a function\\
  (= a program) which takes a proof of $A$ and returns a proof of $B$.
  \bigskip

  Similarly, a proof of $\forall x, P(x)$, is encoded as a function which
  takes $x$ and returns a proof of $P(x)$.
\end{block}


\end{frame}

\begin{frame}{Type theory based proof assistants}

  Proof assistant = a system that:
    \pause\medskip

  \begin{itemize}
  \item \textbf{manipulates} (stores, executes, \dots) functions
    \hfill ($\Lambda$-calculus)
    \medskip

  \item \textbf{checks judgments} such as $a : A$\hfill(typed $\Lambda$-calculus)
  \end{itemize}
  \pause\medskip

  To make it more usable, we need also
  \begin{itemize}
  \item \textbf{building blocks} for custom data structures: \textbf{records,
      unions} \\
    \hfill (Calculus of Inductive Construction $\approx$ Galina)

  \item \textbf{helpers} for writing proof/programs incrementally \\
    \hfill (tactic language).
  \end{itemize}
\end{frame}

\begin{frame}

  You only need to remember:
  \begin{block}{Summary}
    \begin{itemize}
    \item proof, statement, data, programs, etc are all the same first class
      manipulated objects called \textbf{terms}
    \item some particular terms are allowed (from the logic or by their
      definition) to appear on the right of the judgment symbol ``$:$''. They
      are called \textbf{types}
    \item every term has a type
    \end{itemize}

  \end{block}
  \pause

  \Huge\bf

  Enough for the theory\dots
  \bigskip

  \hspace{2cm}Time for a demo\dots
\end{frame}

\section{The Little-Richardson rule}

\section{A few excerpts}

\begin{frame}{Getting definition right}

There is not a single ``good'' definition.
\end{frame}

\section{Should you try ?}

\begin{frame}{Should you try ? My two cents}
  
\end{frame}
\end{document}
% "latex -synctex=1 -shell-escape"
%%% Local Variables: LaTeX-command
%%% mode: latex
%%% TeX-master: t
%%% End: 
