\documentclass[compress,11pt]{beamer}

\usetheme{TALK}
\usepackage[vcentermath]{genyoungtabtikz}
\usepackage{minted}
\usemintedstyle{emacs}
%\usemintedstyle{colorful}
%\usemintedstyle{borland}
%\usemintedstyle{autumn}

\usepackage{genyoungtabtikz}

\newminted{coq}{
frame=lines,
framesep=2mm,
fontsize=\scriptsize,
mathescape=true
}
\usepackage{commath}

% INFO DOCUMENT - TITRE, AUTEUR, INSTITUTION
\title{\bf\LARGE Algebraic Combinatorics using Coq/Mathcomp\\[5mm]}
\subtitle{Dedicated to the memory of Alain Lascoux}
\author{Florent Hivert}
\institute[LRI]{
  LRI / Université Paris Sud 11 / CNRS}
\date{Dagstuhl 18341, August 2018}

\newcommand{\XX}{{\mathbb X}}

\newcommand{\free}[1]{\left\langle#1\right\rangle}
\newcommand{\N}{{\mathbb N}}
\newcommand{\C}{{\mathbb C}}
\newcommand{\Q}{{\mathbb Q}}
\newcommand{\SG}{{\mathfrak S}}
\newcommand{\std}{\operatorname{Std}}

\newcommand{\sym}{\mathrm{sym}}
\newcommand{\NCSF}{\mathbf{NCSF}}
\newcommand{\QSym}{\mathrm{QSym}}
\newcommand{\FSym}{\mathbf{FSym}}

\newcommand{\partof}{\vdash}                    % Partition de
\newcommand{\compof}{\vDash}                    % Composisition de

\newcommand{\qandq}{\text{\quad et\quad}}

\newcommand{\alphX}{{\mathbb X}}
%%%%%%%%%%%%%%%%%%%%%
\renewcommand{\emph}[1]{{\color{red} #1}}
%------------------------------------------------------------------------------
\begin{document}

% PAGE D'ACCUEIL
\frame{\titlepage}

\begin{frame}{Outline}

  \tableofcontents
\end{frame}

\begin{frame}{Algebraic Combinatorics}

\Large
Going back and forth between
\begin{itemize}
\item {\color{red}algebraic identities}
\item {\color{blue}algorithms}
\end{itemize}
\pause\bigskip

Today: Proving a {\color{red}multiplication rule} of symmetric polynomials by
{\color{blue}executing} the Robinson-Schensted algorithm on the
{\color{blue}concatenation of two words}.  \pause\bigskip

\end{frame}

\section{Motivation: Symmetric Polynomials and applications}
\begin{frame}{Symmetric Polynomials}
  
  $n$-variables : $\XX_n := \{x_0, x_1, \dots x_{n-1}\}$.

  Polynomials in $\XX$ : $\C[\XX] = \C[x_0, x_1, \dots, x_{n-1}]$; ex: $3x_0^3x_2
  + 5 x_1x_2^4$.

  \begin{DEFN}[Symmetric polynomial]
    A polynomial is \emph{symmetric} if it is invariant under any permutation of the
    variables: for all $\sigma\in\SG_n$,
    \[P(x_0, x_1, \dots, x_{n-1}) = 
    P(x_{\sigma(0)}, x_{\sigma(1)}, \dots, x_{\sigma({n-1})})\]
  \end{DEFN}

  \[P(a,b,c) = a^2b + a^2c + b^2c + ab^2 + ac^2 + bc^2\]
  \[Q(a,b,c) = 5abc + 3a^2bc + 3ab^2c + 3abc^2\]

\end{frame}

\begin{frame}[fragile]{Integer Partitions}

  different ways of decomposing an integer $n\in\N$ as a sum:
  \[ 5=5=4+1=3+2=3+1+1=2+2+1=2+1+1+1=1+1+1+1+1 \]

  Partition $\lambda := (\lambda_0\geq\lambda_1\geq\dots\geq\lambda_l > 0)$.\\
  $|\lambda| := \lambda_0+\lambda_1+\dots+\lambda_l \qandq
  \ell(\lambda) := l\,. $

  Ferrer's diagram of a partitions : $(5,3,2,2) \leftrightarrow \yngs(0.5, 2,2,3,5)$

% \end{frame}

% \begin{frame}[fragile]{Integer Partitions}

% Formal definition:
\begin{coqcode}
  Fixpoint is_part sh := (* Predicate *)
    if sh is sh0 :: sh'
    then (sh0 >= head 1 sh') && (is_part sh')
    else true.

  (* Boolean reflection lemma *)
  Lemma is_partP sh : reflect
    (last 1 sh != 0 /\ forall i, (nth 0 sh i) >= (nth 0 sh i.+1))
    (is_part sh).

  Lemma is_part_sortedE sh : (is_part sh) = (sorted geq sh) && (0 \notin sh).
\end{coqcode}

\end{frame}

% \begin{frame}{Monomial symmetric polynomials}

%   Partition $\lambda := (\lambda_0\geq\lambda_1\geq\dots\geq\lambda_l > 0)$.\\
%   \[ |\lambda| := \lambda_0+\lambda_1+\dots+\lambda_l \qandq
%   \ell(\lambda) := l\,. \]

%   \begin{DEFN}[Monomial symmetric polynomial]
%     sum of all the distinct permutation of a monomial
%   \end{DEFN}
%   \[m_{(2,1)}(a,b,c) = a^2b + a^2c + b^2c + ab^2 + ac^2 + bc^2\]
%   \[m_{(2,1,1)}(a,b,c) = 3a^2bc + 3ab^2c + 3abc^2\]
%   \begin{PROP}
%     $(m_{\lambda}(\XX_n))_\lambda$ where $\lambda$ is a \emph{partition} of length
%     less than $n$ is a \emph{basis} of the algebra of symmetric polynomials.
%   \end{PROP}
% \end{frame}

% \begin{frame}{Antisymmetric polynomials}

%   \begin{DEFN}[Antiymmetric polynomial]
%     A polynomial is \emph{antisymmetric} if for all $\sigma\in\SG_n$,
%     \[P(x_0, x_1, \dots, x_{n-1}) = \operatorname{sign}(\sigma)
%     P(x_{\sigma(0)}, x_{\sigma(1)}, \dots, x_{\sigma({n-1})})\]
%     Equivalently for any $0\leq i < j < n$
%     \[
%     P(x_{0},\dots,x_i,\dots,x_j,\dots,x_{n-1})
%     = -
%     P(x_{0},\dots,x_j,\dots,x_i,\dots,x_{n-1})
%     \]
%   \end{DEFN}
%   \begin{REMARK}
%     $P(a,b) = - P(b,a)$ implies that $P(a,a)=0$ and therefore $P$ is divisible by $(b-a)$.
%   \end{REMARK}
% \end{frame}

\begin{frame}{Schur symmetric polynomials (Jacobi)}

  \begin{DEFN}[Schur symmetric polynomial]
    \small Partition $\lambda :=
    (\lambda_0\geq\lambda_1\geq\dots\geq\lambda_{l-1})$ with $l\leq
    n$; set $\lambda_i:=0$ for $i\geq l$.
    \[
    s_{\lambda} = 
    \frac{\sum_{\sigma\in\SG_n} \operatorname{sign}(\sigma)
      \XX_n^{\sigma(\lambda+\rho)}}%
    {\prod_{0\leq i<j<n} (x_j - x_i)}
    = \frac{\left|
       \begin{smallmatrix}
         x_1^{\lambda_{n-1}+0}  & x_2^{\lambda_{n-1}+0}   & \dots  & x_n^{\lambda_{n-1}+0}  \\
         x_1^{\lambda_{n-2}+1}  & x_2^{\lambda_{n-2}+1}   & \dots  & x_n^{\lambda_{n-2}+1}  \\
         \vdots & \vdots & \ddots & \vdots \\
         x_1^{\lambda_1+n-2}  & x_2^{\lambda_1+n-2}   & \dots  & x_n^{\lambda_1+n-2}  \\
         x_1^{\lambda_0+n-1}  & x_2^{\lambda_0+n-1}   & \dots  & x_n^{\lambda_0+n-1}  \\
      \end{smallmatrix}
      \right|
    }{\left|
       \begin{smallmatrix}
         1      & 1      & \dots  & 1     \\
         x_1    & x_2    & \dots  & x_n    \\
         x_1^2  & x_2^2   & \dots  & x_n^2  \\
         \vdots & \vdots & \ddots & \vdots \\
         x_1^{n-1}  & x_2^{n-1}   & \dots  & x_n^{n-1}  \\
      \end{smallmatrix}
      \right|
    }
    \]
  \end{DEFN}
  \[s_{(2,1)}(a,b,c) = a^2b + ab^2 + a^2c + 2abc + b^2c + ac^2 + bc^2\]
\end{frame}


\begin{frame}{Littlewood-Richardson coefficients}

  \begin{PROP}
    The family $(s_\lambda(\XX_n))_{\ell(\lambda) \leq n}$ is a (linear) basis of the
    ring of symmetric polynomials on $\XX_n$.
  \end{PROP}

  \begin{DEFN}[Littlewood-Richardson coefficients]
    Coefficients $c_{\lambda,\mu}^\nu$ of the expansion of the product:
    \[
    s_\lambda s_\mu = \sum_{\nu} c_{\lambda,\mu}^\nu s_\nu\,.
    \]
  \end{DEFN}
  Fact: $s_\lambda(\XX_{n-1}, x_n := 0) =
  s_\lambda(\XX_{n-1}) \text{ if } \ell(\lambda) < n\text{ else } 0$.

  Consequence: $c_{\lambda,\mu}^\nu$ are independant of the number of variables.
\end{frame}

\begin{frame}{Quotes}

  The Littlewood–Richardson rule is notorious for the number of errors that
  appeared prior to its complete, published proof. Several published attempts
  to prove it are incomplete, and it is particularly difficult to avoid errors
  when doing hand calculations with it: even the original example in
  D.~E.~Littlewood and A.~R.~Richardson (1934) contains an error --
  Wikipedia  \bigskip

  Unfortunately the Littlewood-Richardson rule is much harder to prove than
  was at first suspected. I was once told that the Littlewood-Richardson rule
  helped to get men on the moon but was not proved until after they got
  there. The first part of this story might be an exaggeration. -– Gordon
  James

\end{frame}

\begin{frame}{Sample of applications}
  \begin{itemize}
  \item $\#P$-complete problem.
  \item Mulmuley's geometric complexity theory: possibly $P \neq NP$
  \item multiplicity of induction or restriction of irreducible representations
    of the symmetric groups;
  \item multiplicity of the tensor product of the irreducible representations
    of linear groups;
  \item Geometry: mumber of intersection in a grassmanian variety, cup product
    of the cohomology;
  \item Horn problem: eigenvalues of the sum of two hermitian matrix;
  \item Extension of abelian groups (Hall algebra);
  \item Application in quantum physics (spectrum rays of the Hydrogen atoms);
  \end{itemize}
\end{frame}

\section{Combinatorial ingredients: partitions and tableaux}

\Yboxdim{10pt}
\begin{frame}{Young Tableau}

  \begin{DEFN}
    \begin{itemize}
    \item Filling of a partition shape;
    \item non decreasing along the rows;
    \item strictly increasing along the columns.
      \medskip
    \item row reading
    \end{itemize}
  \end{DEFN}
  \[\scriptsize
  \young(dde,bcccd,aaabbde) = ddebcccdaaabbde
  \qquad\qquad
  \young(5,269,13478)=526913478\]

\end{frame}

\begin{frame}[fragile]{Young Tableau: formal definition}

  Using \texttt{SSReflect} class/mixin/canonical paradigm (Work in
  progress: merge with C.~Cohen order.
\begin{coqcode}
  Variable (T : ordType) (Z : T).

  Definition dominate (u v : seq T) :=
    ((size u) <= (size v)) &&
     (all (fun i => (nth Z u i > nth Z v i)%Ord) (iota 0 (size u))).

  Lemma dominateP u v :
    reflect ((size u) <= (size v) /\
             forall i, i < size u -> (nth Z u i > nth Z v i)%Ord)
            (dominate u v).

  Fixpoint is_tableau (t : seq (seq T)) :=
    if t is t0 :: t' then
      [&& (t0 != [::]), sorted t0,
        dominate (head [::] t') t0 & is_tableau t']
    else true.

  Definition to_word t := flatten (rev t).
\end{coqcode}
\end{frame}

\begin{frame}{Combinatorial definition of Schur functions}

  \begin{DEFN}
    \[s_\lambda(\XX) = \sum_\text{$T$ tableaux of shape $\lambda$} \XX^T\]
    where $\XX^T$ is the product of the elements of $T$.
  \end{DEFN}
  \[s_{(2,1)}(a,b,c) = a^2b\,+\,ab^2\,+\,a^2c\ +\quad 2abc\quad +\
  b^2c\,+\,ac^2\,+\,bc^2\]
  \[s_{(2,1)}(a,b,c) = {  \Yboxdim{8pt}\scriptsize
    \young(b,aa) + \young(b,ab) + \young(c,aa)
  + \young(b,ac) + \young(c,ab) + \young(c,bb) + \young(c,ac) +
  \young(c,bc)}\]
  \pause\medskip

  Note: I'll prove the equivalence of the two definitions as a
  \emph{consequence of a particular case of the LR-rule} by relating it with
  recursively unfolding determinants.
\end{frame}

\begin{frame}[fragile]
\begin{coqcode}
Variable n : nat.
Variable R : comRingType.

(* {mpoly R[n]} : the ring of polynomial over the commutative ring R *)
(*                in n variables (P.-Y. Strub)                     *)

Definition is_tab_of_shape (sh : seq (seq 'I_n)) :=
  [ pred t | (is_tableau t) && (shape t == sh) ].

Structure tabsh := TabSh {tabshval; _ : is_tab_of_shape sh tabshval}.
[...]
Canonical tabsh_finType := [...]

Definition Schur d (sh : 'P_d) : {mpoly R[n]} :=
  \sum_(t : tabsh n sh) \prod_(v <- to_word t) 'X_v.
\end{coqcode}
\end{frame}

\begin{frame}{Yamanouchi Words}

  $\abs{w}_x$ : number of occurrence of $x$ in $w$.

  \begin{DEFN}
    Sequence $w_0,\dots,w_{l-1}$ of integers such that for all $k, i$,
    \[ \abs{w_i,\dots,w_{l-1}}_k \geq \abs{w_i,\dots,w_{l-1}}_{k+1} \]

    Equivalently $(\abs{w}_i)_{i\leq\max(w)}$ is a partition and $w_1,\dots,w_{l-1}$ is
    also Yamanouchi.
  \end{DEFN}

  \[ (), 0, 00, 10, 000, 100, 010, 210, \]
  \[ 0000, 1010, 1100, 0010, 0100, 1000, 0210, 2010, 2100, 3210 \]
\end{frame}

\section{The rule}

\begin{frame}{The LR Rule at last !}

  \begin{THEO}[Littlewood-Richardson rule]
    $c_{\lambda, \mu}^{\nu}$ is the number of (skew) tableaux of shape the
    difference $\nu/\lambda$, whose row reading is a Yamanouchi word of
    evaluation $\mu$.
  \end{THEO}
% ...00 ...00 ...00
% ...1  ...1  ...1 
% .00   .01   .02  
% 12    02    01   
  \[
  C_{331,421}^{5432} = 3
  \qquad \Yboxdim{7pt}\tiny
  \gyoung(12,:;00,:::;1,:::;00)\qquad
  \gyoung(02,:;01,:::;1,:::;00)\qquad
  \gyoung(01,:;02,:::;1,:::;00)
  \]
% ...00 ...00 ...00 ...00 ...00 ...00 ...00 ...00 ...00 ...00 ...00 ...00 ...00 ...00 ...00
% ...1  ...1  ...1  ...1  ...1  ...1  ...1  ...1  ...1  ...1  ...1  ...1  ...1  ...1  ...1
% .00   .00   .00   .01   .00   .01   .02   .02   .12   .02   .02   .01   .01   .12   .12
% 01    02    11    01    12    02    01    13    03    03    11    12    22    02    23
  \[
  C_{431,4321}^{7542} = 4
  \qquad \Yboxdim{7pt}\tiny
  \gyoung(23,:;112,:::;01,::::;000)\quad
  \gyoung(23,:;012,:::;11,::::;000)\quad
  \gyoung(13,:;022,:::;11,::::;000)\quad
  \gyoung(03,:;122,:::;11,::::;000)\quad
  \]
  \[
  C_{4321,431}^{7542} = 4
  \qquad \Yboxdim{7pt}\tiny
  \gyoung(:;2,::;11,:::;01,::::;000)\quad
  \gyoung(:;1,::;12,:::;01,::::;000)\quad
  \gyoung(:;1,::;02,:::;11,::::;000)\quad
  \gyoung(:;0,::;12,:::;11,::::;000)\quad
  \]

\end{frame}

\begin{frame}[fragile]{The formal statement}

\begin{coqcode}
Definition is_skew_reshape_tableau (P P1 : seq nat) (w : seq nat) :=
  is_skew_tableau P1 (skew_reshape P1 P w).
Lemma is_skew_reshape_tableauP (w : seq nat) :
  size w = sumn (P / P1) ->
  reflect
    (exists tab, [/\ is_skew_tableau P1 tab,
                     shape tab = P / P1 & to_word tab = w])
    (is_skew_reshape_tableau P P1 w).

Definition LRyam_set :=
  [set y : yameval P2 | is_skew_reshape_tableau P P1 y].
Definition LRyam_coeff := #|LRyam_set|.
\end{coqcode}
Then
\begin{coqcode}
Theorem LRyam_coeffP :
  Schur P1 * Schur P2 =
  \sum_(P : 'P_(d1 + d2) | included P1 P) Schur P *+ LRyam_coeff P.
\end{coqcode} 

\end{frame}

\begin{frame}{Idea of the proof}

  \begin{itemize}
  \item The longest increasing subsequence problem
  \item Robinson-Schensted (RS) bijection:
    \[
    ababcabbad\quad \longleftrightarrow\quad \young(c,bbb,aaaabd),
    \young(8,256,013479)
    \]
  \item Reimplement RS using some \emph{local rewriting rules}
  \item Realize that RS is actually computing a
    \emph{normal form in a quotient of the free monoid}
  \item lift the LR rule at a \emph{non-commutative} level
  \item The proof is done by \emph{symbolically executing the
      RS algorithms on the concatenation of two words}.
  \end{itemize}
\end{frame}

\begin{frame}{Outline of a proof}

  \textsc{\sc Lascoux, Leclerc and Thibon} \textit{The Plactic monoid},
    {\it in} M.~Lothaire, Algebraic combinatorics on words,
    Cambridge Univ. Press.

  \begin{enumerate}
  \item increasing subsequences and Schensted's algorithms;
  \item Robinson-Schensted correspondance : a bijection;
  \item Green's invariants: computing the maximum sum of the length of $k$ disjoint
    non-decreassing subsequences;
  \item Knuth relations, the plactic monoïd;
  \item Green's invariants are plactic invariants: Equivalence between RS and plactic;
  \item standardization; symmetry of RS;
  \item Lifting to non commutative polynomials : Free quasi-symmetric function
    and shuffle product;
  \item non-commutative lifting the LR-rule : The free/tableau LR-rule;
  \item Back to Yamanouchi words: a final bijection.
  \end{enumerate}
\end{frame}


\section{The longest increasing subsequence problem}

\newcommand{\red}[1]{{\color{red} #1}}
\newcommand{\grn}[1]{{\color{green} #1}}
\newcommand{\blu}[1]{{\color{blue} #1}}
\begin{frame}{The longest increasing subsequence problem}

  Some increasing subsequences: 
  \[
  \begin{array}{c@{}c@{}c@{}c@{}c@{}c@{}c@{}c@{}c@{}c@{}c@{}c@{}c}
    a&b&a&b&c&a&b&b&a&d&b&a&b\\
    a&b&\red{a}&b&c&\red a&\red b&\red b&a&\red d&b&a&b \\
    \red a&\red b&a&\red{b}&c&a&\red b&\red b&a&d&\red b&a&\red b
  \end{array}
  \]
  \begin{PROBLEM}[Schensted]
    Given a finite sequence $w$, compute the maximum length of a increasing subsequence.
  \end{PROBLEM}
\end{frame}

\newcommand{\ar}[1]{\xrightarrow{#1}}

\begin{frame}[fragile]{Schensted's algorithm}

  \begin{ALGO}
    Start with an empty row $r$, insert the letters $l$ of the word one by one
    from left to right by the following rule:
    \begin{itemize}
    \item replace the first letter strictly larger that $l$ by $l$;
    \item append $l$ to $r$ if there is no such letter.
    \end{itemize}
  \end{ALGO}
  \bigskip

  Insertion of $\begin{array}{c@{}c@{}c@{}c@{}c@{}c@{}c@{}c@{}c@{}c@{}c@{}c@{}c}
    a&b&a&b&c&a&b&b&a&d&b&a&b\\
  \end{array}w$
  \begin{multline*}
  \emptyset\ar{a}\young(a)\ar{b}\young(ab)\ar{a}\young(aa)\ar{b}\young(aab)
  \ar{c}\young(aabc)\ar{a}\\
  \young(aaac)\ar{b}\young(aaab)\ar{b}\young(aaabb)\ar{a}\\\young(aaaab)\ar{d}
  \young(aaaabd)\ar{b}\young(aaaabb)\ar{a}\\
  \young(aaaaab)\ar{b}\young(aaaaabb)
  \end{multline*}
\end{frame}

\begin{frame}[fragile]{Schensted's specification}

  \emph{Warning}: list index start from $0$.
  \bigskip

  \begin{THEO}[Schensted 1961]
    The $i$-th entry $r[i+1]$ of the row $r$ is the smallest letter which ends
    a increasing subsequence of length $i$.
  \end{THEO}
\[
\operatorname{Schensted}(
\begin{array}{c@{}c@{}c@{}c@{}c@{}c@{}c@{}c@{}c@{}c@{}c@{}c@{}c}
  a&b&a&b&c&a&b&b&a&d&b&a&b\\
\end{array})
= \young(aaaaabb)
\]
\end{frame}


\begin{frame}[fragile]

  \begin{coqcode}
Fixpoint insrow r l : seq T :=
  if r is l0 :: r then
    if (l < l0)%Ord then l :: r
    else l0 :: (insrow r l)
  else [:: l].

Fixpoint inspos r (l : T) : nat :=
  if r is l0 :: r' then
    if (l < l0)%Ord then 0
    else (inspos r' l).+1
  else 0.

Definition ins r l := set_nth l r (inspos r l) l.

Lemma insE r l : insmin r l = ins r l.
Lemma insrowE r l : insmin r l = insrow r l.
  \end{coqcode}
\end{frame}

\begin{frame}[fragile]
  \begin{coqcode}
  (* rev == list reversal,   rcons s x == the sequence s, followed by x *)
  (* subseq s1 s2 == s1 is a subsequence of s2                          *)
  Fixpoint Sch_rev w := if w is l0 :: w' then ins (Sch_rev w') l0 else [::].
  Definition Sch w := Sch_rev (rev w).

  Lemma is_row_Sch w : is_row (Sch w).

  Definition subseqrow s w := subseq s w && is_row s.
  Definition subseqrow_n s w n := [&& subseq s w , (size s == n) & is_row s].

  Theorem Sch_exists w i :
    i < size (Sch w) ->
    exists s : seq T, (last Z s == nth Z (Sch w) i) && subseqrow_n s w i.+1.
  (* Induction elimining the last letter : elim/last_ind: w *)
  Theorem Sch_leq_last w s si :
    subseqrow (rcons s si) w ->
    size s < size (Sch w) /\ (nth Z (Sch w) (size s) <= si)%Ord.
  (* Induction elimining the last letter : elim/last_ind: w *)

  Theorem Sch_max_size w :
    size (Sch w) = \max_(s : subseqs w | is_row s) size s.
  \end{coqcode}
\end{frame}

\begin{frame}[fragile]{The Robinson-Schensted's correspondence}

  {\color{blue}Bumping the letters}: when a letter is replaced in Schensted
  algorithm, \textbf{insert it in a next row} (placed on top in the drawing).
  \newcommand{\ra}{{\red a}}%
  \newcommand{\rb}{{\red b}}%
  \newcommand{\rc}{{\red c}}%
  \newcommand{\rd}{{\red d}}%
  \begin{multline*}
  \emptyset\ar{a}\young(\ra )\ar{b}\young(a\rb)\ar{a}\young(\rb,a\ra)\ar{b}\young(b,aa\rb)
  \ar{c}\young(b,aab\rc)\ar{a}\\[3mm]
  \young(b\rb,aa\ra c)\ar{b}\young(bb\rc,aaa\rb)\ar{b}\young(bbc,aaab\rb)\ar{a}\\[3mm]
  \young(\rc,bb\rb,aaa\ra b)\ar{d}
  \young(c,bbb,aaaab\rd)\ar{b}\young(c,bbb\rd,aaaab\rb)\ar{a}\\[3mm]
  \young(c\rd,bbb\rb,aaaa\ra b)\ar{b}\young(cd,bbbb,aaaaab\rb)
  \end{multline*}
\end{frame}


\begin{frame}[fragile]{Going back}

  If we remember which cell was added we can recover the letter and the
  previous tableau:
  \newcommand{\ra}{{\red a}}%
  \newcommand{\rb}{{\red b}}%
  \newcommand{\rc}{{\red c}}%
  \newcommand{\rd}{{\red d}}%
  \[
  \only<1>{\yng(1,4,6)\ar{?} \young(cd,bbbb,aaaaab)}
  \only<2>{\young(c,bbbd,aaaabb)\ar{a} \young(c\rd,bbb\rb,aaaa\ra b)}
  \]
\pause

  \begin{coqcode}
RSmap : forall T : ordType, seq T -> seq (seq T) * seq nat
RSmapinv2 : forall T : ordType, seq (seq T) * seq nat -> seq T

Definition is_RSpair pair := let: (P, Q) := pair
  in [&& is_tableau P, is_yam Q & (shape P == shape_rowseq Q)].
Theorem RSmap_spec w : is_RSpair (RSmap w).

Theorem RS_bij_1 w : RSmapinv2 (RSmap w) = w.
Theorem RS_bij_2 pair : is_RSpair pair -> RSmap (RSmapinv2 pair) = pair.
  \end{coqcode}
\end{frame}


% \begin{frame}[fragile]{Robinson-Schensted's bijection (Yamanouchi version)}

%   \newcommand{\ra}{{\red a}}%
%   \newcommand{\rb}{{\red b}}%
%   \newcommand{\rc}{{\red c}}%
%   \newcommand{\rd}{{\red d}}%
%   \begin{multline*}
%   \emptyset,\emptyset \ar{a}
%   \young(\ra ),         {\tiny 0}\ar{b}
%   \young(a\rb),         00\ar{a}
%   \young(\rb,a\ra),     100\ar{b}\\[3mm]
%   \young(b,aa\rb),      0100\ar{c}
%   \young(b,aab\rc),     00100\ar{a}
%   \young(b\rb,aa\ra c), 100100\ar{b}\\[3mm]
%   \young(bb\rc,aaa\rb), 1100100\ar{b}
%   \young(bbc,aaab\rb),  01100100\ar{a}\\[3mm]
%   \young(\rc,bb\rb,aaa\ra b), 201100100\ar{d}
%   \young(c,bbb,aaaab\rd), 0201100100
%   \end{multline*}
% \end{frame}

\begin{frame}[fragile]{Robinson-Schensted's bijection (Tableau version)}

  \newcommand{\ra}{{\red a}}%
  \newcommand{\rb}{{\red b}}%
  \newcommand{\rc}{{\red c}}%
  \newcommand{\rd}{{\red d}}%
  \begin{multline*}
  \emptyset,\emptyset \ar{a}
  \young(\ra ),         \young(0)\ar{b}
  \young(a\rb),         \young(01)\ar{a}
  \young(\rb,a\ra),     \young(2,01)\ar{b}\\[3mm]
  \young(b,aa\rb),      \young(2,013)\ar{c}
  \young(b,aab\rc),     \young(2,0134)\ar{a}
  \young(b\rb,aa\ra c), \young(25,0134)\ar{b}\\[3mm]
  \young(bb\rc,aaa\rb), \young(256,0134)\ar{b}
  \young(bbc,aaab\rb),  \young(256,01347)\ar{a}\\[3mm]
  \young(\rc,bb\rb,aaa\ra b), \young(8,256,01347)\ar{d}
  \young(c,bbb,aaaab\rd), \young(8,256,013479)
  \end{multline*}
\end{frame}

\begin{frame}[fragile]{Idea of the proof: the non commutative lifting}

  Fix the second tableau $Q$ in the bijection
  \[L_Q := \{ w\ \mid\ RS(w)_2 = Q \}\]
  Then clearly:
    \[S_{\operatorname{shape}(Q)} = \sum_{w\in L_Q} \operatorname{comm}(w)\]
    \pause\bigskip
  The crucial fact is:
  \begin{THEO}
    There exists an explicit set $\Omega(Q, R)$ such that
    \[L_Q L_R = \sum_{T\in\Omega(Q, R)} L_T.\]
  \end{THEO}
\end{frame}

\section{Further developments and conclusion}
\begin{frame}[fragile]{Statement in group theory}

  Frobenius defined an isometry from symmetric function to the character ring
  of the symmetric groups. 

  We can emph{translate the statement into a group theoretic statement}:
  \begin{coqcode}
'SG_n      : the symmetric groups on the set [0, .., n-1]
permCT mu  : an element of the conjugacy class indexed by mu
'irrSG[la] : the irreducible character for ['SG_n] associated to the
             partition [la] of n.

Theorem Frobenius_char n (la mu : 'P_n) :
  'irrSG[la] (permCT mu) = (Delta * \prod_(d <- mu) P d)@_(mpart la + rho n).

Theorem LR_rule_irrSG c d (la : 'P_c) (mu : 'P_d) :
  'Ind['SG_(c + d)] ('irrSG[la] \o^ 'irrSG[mu]) =
  \sum_(nu : 'P_(c + d) | included la nu) 'irrSG[nu] *+ LRyam_coeff la mu nu.
  \end{coqcode}

\end{frame}


\begin{frame}{Conclusion}

  \begin{center}
    \emph{\huge\bf It's feasible !!!!}
  \end{center}
  \bigskip

  even by someone without any prior knowledge of type theory or lambda
  calculus.
  \bigskip  \bigskip\pause

  Schützenberger's proof was correct !
  \bigskip\pause

  A certified implementation ($\#P$-complete)

\end{frame}

\begin{frame}{Easy / Hard points}

  \begin{itemize}
  \item Thanks to boolean reflexion SSReflect/Mathcomp is very good at
    automatically dealing with trivial cases (size $*=\frac{1}{3}$); \medskip

  \item A small number of missing basic lemmas;
    \pause

  \item Hard to expand the class/mixin/canonical paradigm;
    \medskip

  \item Easy to interface algebra with combinatorics;

  \item Tuple and dependant types (the position of a letter in the word w);
    \medskip

  \item More generally, I feel that Mathcomp is too much oriented toward
    finite;
    \medskip

  \item Dealing with a lot of hypothesis.
  \end{itemize}
  
\end{frame}
\end{document}
% "latex -synctex=1 -shell-escape"
%%% Local Variables: LaTeX-command
%%% mode: latex
%%% TeX-master: t
%%% End: 
