\documentclass{article}

\usepackage[vcentermath,enableskew]{youngtab}
\usepackage{minted}
\usemintedstyle{emacs}
%\usemintedstyle{colorful}
%\usemintedstyle{borland}
%\usemintedstyle{autumn}

\newminted{coq}{
frame=lines,
framesep=2mm,
fontsize=\scriptsize,
mathescape=true
}
\usepackage{commath}

% INFO DOCUMENT - TITRE, AUTEUR, INSTITUTION
\title{\bf\LARGE A certified implementation of \\
Littlewood-Richardson rule\\[5mm]}
\author{Jean-Christophe Filliatre and Florent Hivert}
%\institute[LRI]{
%  LRI / Université Paris Sud 11 / CNRS / INRIA}
\date{Mai 2015}

\newcommand{\XX}{{\mathbb X}}

\newcommand{\exec}{\operatorname{exec}}
\newcommand{\free}[1]{\left\langle#1\right\rangle}
\newcommand{\gl}{{\mathfrak gl}}
\newcommand{\bb}{\mathbb}
\newcommand{\A}{{\mathbb A}}
\newcommand{\E}{{\mathbb E}}
\newcommand{\F}{\mathbf{F}}
\newcommand{\GG}{\mathbf{G}}
\newcommand{\G}{\mathbf{G}}
\newcommand{\N}{{\mathbb N}}
\newcommand{\C}{{\mathbb C}}
\newcommand{\R}{{\mathbb R}}
\newcommand{\Z}{{\mathbb Z}}
\newcommand{\Q}{{\mathbb Q}}
\newcommand{\SG}{{\mathfrak S}}
\newcommand{\ft}{\tilde{f}}
\newcommand{\et}{\tilde{e}}
\newcommand{\std}{\operatorname{Std}}
\newcommand{\Des}{\operatorname{Des}}
\newcommand{\Rec}{\operatorname{Rec}}
\newcommand{\Sh}{\operatorname{Sh}}

\newcommand{\sym}{\mathrm{sym}}
\newcommand{\NCSF}{\mathbf{NCSF}}
\newcommand{\QSym}{\mathrm{QSym}}
\newcommand{\FSym}{\mathbf{FSym}}
\newcommand{\FQSym}{\mathbf{FQSym}}
\newcommand{\MQSym}{\mathbf{MQSym}}
\newcommand{\Tree}{\mathbf{Tree}}

\newcommand{\MS}{\mathbf{MS}}
\newcommand{\FS}{{\mathbb \phi S}}
\newcommand{\SSs}{\mathbb{S}}
\newcommand{\TT}{{\cal T}}
\newcommand{\PP}{{\cal P}}
\def\Pp{{\bf P}}        % P de PBT
\def\Qq{{\bf Q}}        % Q de PBT^*
\newcommand{\HQ}{{\sf Q}}
\newcommand{\HR}{{\sf R}}
\newcommand{\rad}{\operatorname{rad}}
\newcommand{\soc}{\operatorname{soc}}
\newcommand{\Ext}{\operatorname{Ext}}
\newcommand{\End}{\operatorname{End}}
\newcommand{\conn}{{\cal C}}
\newcommand{\ff}{{\sf F}}
\newcommand{\UU}{\mathbf{U}}
\newcommand{\VV}{\mathbf{V}}
\newcommand{\sfact}{\operatorname{\rm sfact}}

\newcommand{\partof}{\vdash}                    % Partition de
\newcommand{\compof}{\vDash}                    % Composisition de
\newcommand{\ssh}{\Cup}
\newcommand{\saug}{\uplus}
\newcommand{\sconc}{\bullet}
\newcommand{\Std}{{\rm Std}}
\newcommand{\Park}{{\rm Park}}
\newcommand{\sumord}{\hat +}

\newcommand{\qandq}{\text{\quad et\quad}}

\newcommand{\tensor}{\otimes}
\newcommand{\pairing}[2]{\left\langle#1,#2\right\rangle} % Crochet de dualite

\newcommand{\red}[1]{{\color{red} #1}}
\newcommand{\grn}[1]{{\color{green} #1}}
\newcommand{\blu}[1]{{\color{blue} #1}}

%%%%%%%%%%%%%%%%%%%%%
\newcommand{\alphX}{{\mathbb X}}
\renewcommand{\emph}[1]{{\color{red} #1}}


\newtheorem{THEO}{Theorem}
\newtheorem{PROP}{Proposition}
\newtheorem{LEMMA}{Lemma}
\newtheorem{CORO}{Corollary}
\newtheorem{PROBLEM}{Problem}
\newtheorem{REMARK}{Remark}
\newtheorem{NOTE}{Note}

% \theoremstyle{definition}
\newtheorem{DEFN}{Definition}
\newtheorem{DEFNs}{Definitions}
\newtheorem{ALGO}{Algorithm}


%------------------------------------------------------------------------------
\begin{document}

\maketitle

\section{The rule}

\begin{THEO}[Littlewood-Richardson rule]
  $c_{\lambda, \mu}^{\nu}$ is the number of (skew) tableaux of shape the
  difference $\nu/\lambda$, whose row reading is a Yamanouchi word of
  evaluation $\mu$.
\end{THEO}

% ...00 ...00 ...00
% ...1  ...1  ...1 
% .00   .01   .02  
% 12    02    01   

Some examples:
  \[
  C_{331,421}^{5432} = 3
  \qquad \tiny
  \young(12,:00,:::1,:::00)\qquad
  \young(02,:01,:::1,:::00)\qquad
  \young(01,:02,:::1,:::00)
  \]
% ...00 ...00 ...00 ...00 ...00 ...00 ...00 ...00 ...00 ...00 ...00 ...00 ...00 ...00 ...00
% ...1  ...1  ...1  ...1  ...1  ...1  ...1  ...1  ...1  ...1  ...1  ...1  ...1  ...1  ...1
% .00   .00   .00   .01   .00   .01   .02   .02   .12   .02   .02   .01   .01   .12   .12
% 01    02    11    01    12    02    01    13    03    03    11    12    22    02    23
  \[
  C_{4321,431}^{7542} = 4
  \qquad \tiny
  \young(:2,::11,:::01,::::000)\quad
  \young(:1,::12,:::01,::::000)\quad
  \young(:1,::02,:::11,::::000)\quad
  \young(:0,::12,:::11,::::000)\quad
  \]


  \def\AA{\red 0}
  \def\AB{\grn 1}
  \def\AC{\blu 2}
  \def\AD{{\color{pink} 3}}
  \[
  C_{431,4321}^{7542} = 4
  \qquad \tiny
  \young(\AC\AD,:\AB\AB\AC,:::\AA\AB,::::\AA\AA\AA)\quad
  \young(\AC\AD,:\AA\AB\AC,:::\AB\AB,::::\AA\AA\AA)\quad
  \young(\AB\AD,:\AA\AC\AC,:::\AB\AB,::::\AA\AA\AA)\quad
  \young(\AA\AD,:\AB\AC\AC,:::\AB\AB,::::\AA\AA\AA)\quad
  \]

  \def\AA{\red 0}
  \def\AB{\grn 1}
  \def\AC{\blu 2}
  \def\AD{{\color{pink} 3}}
  \[
  C_{431,4322}^{7543} = 2
  \qquad \tiny
  \young(\AB\AD\AD,:\AA\AC\AC,:::\AB\AB,::::\AA\AA\AA)\quad
  \young(\AA\AD\AD,:\AB\AC\AC,:::\AB\AB,::::\AA\AA\AA)\quad
  \]


\pagebreak
\section{Data structures}
\newcommand{\inn}{\texttt{inner}}
\newcommand{\out}{\texttt{outer}}
\newcommand{\evl}{\texttt{eval}}
\newcommand{\innev}{\texttt{innev}}
\newcommand{\work}{\texttt{work}}
\newcommand{\eor}{\texttt{end\_of\_row}}
\newcommand{\eoo}{\texttt{end\_of\_overlap}}
\newcommand{\below}{\texttt{below}}
First step: adding sentinels, here are the requirement:
\begin{itemize}
\item the first row is empty;
\item the lenght of \texttt{inner} and \texttt{outer} are equal;
\item \texttt{eval} ends with a $0$.
\end{itemize}
For example, $\out = 7543$, $\inn=431$ and
$\evl=4321$ is transformed as $\out=77543$,
$\inn=74310$ and $\evl=43220$.
\[
\young(\ \ \ ,\bullet\ \ \ ,\bullet\bullet\bullet\ \ ,\bullet\bullet\bullet\bullet\ \ \ ,\bullet\bullet\bullet\bullet\bullet\bullet\bullet)
\]
The contents of the box are stored in an array called \work{} in the following order:
\def\Dix{10}
\[
\young(\Dix98,\bullet765,\bullet\bullet\bullet43,\bullet\bullet\bullet\bullet210,\bullet\bullet\bullet\bullet\bullet\bullet\bullet)
\]
For example
\[  \young(\AA\AD\AD,:\AB\AC\AC,:::\AB\AB,::::\AA\AA\AA) \]
is stored in reverse order as
\[
\begin{array}{|c|c|c||c|c||c|c|c||c|c|c|}
  \hline
  0   & 1   & 2   & 3   & 4   & 5   & 6   & 7   & 8   & 9   & 10  \\
  \hline
  \AA & \AA & \AA & \AB & \AB & \AC & \AC & \AB & \AD & \AD & \AA \\
  \hline
\end{array}
\]
Or writing the array in reverse order:
\[
\begin{array}{|c|c|c||c|c|c||c|c||c|c|c|}
  \hline
  10  & 9   & 8   & 7   & 6   & 5   & 4   & 3   & 2   & 1   & 0 \\
  \hline
  \AA & \AD & \AD & \AB & \AC & \AC & \AB & \AB & \AA & \AA & \AA \\
  \hline
\end{array}
\]
We use some index arrays:
\[
  \begin{array}{|c|c|c|c|c|c|c|}
    \hline
    & 0 & 1 & 2 & 3 & 4 & \\
    \hline
    \hline
    \out & 7 & 7 & 5 & 4 & 3 & \\
    \hline
    \inn & 7 & 4 & 3 & 1 & 0 & \\
    \hline
    \hline
    \eor & 0 & 3 & 5 & 8 & 11 & \sum_{j=1}^i  \out[j] - \inn[j]\\
    \hline
    \below & ? & 0 & 1 & 1 & 2 & \out[i] - \inn[i-1]\\
    \hline
    \eoo & ? & 0 & 4 & 6 & 10 & \eor[i-1] + \below[i]\\
    \hline
  \end{array}
\]



\end{document}
% "latex -synctex=1 -shell-escape"
%%% Local Variables: LaTeX-command
%%% mode: latex
%%% TeX-master: t
%%% End: 
